% Purpose: an example .tex file for the thesis.

\documentclass[12pt,leqno]{report}

% citation style can be whatever is "accepted in your field"
\usepackage[round]{natbib}

% SUNY NEW PALTZ thesis
% include ur_thesis after citation style to ensure bibliography in toc
\usepackage{ur_thesis}

\usepackage{times}
\usepackage{hyperref}
\usepackage{xcolor}		% make links dark blue
\definecolor{darkblue}{rgb}{0, 0, 0.5}
\hypersetup{colorlinks=true,citecolor=darkblue, linkcolor=darkblue, urlcolor=darkblue}
  
\begin{document}

\sloppy
\title{LDR Data Augmentation for Convolutional Neural Network Construction}
\author{Michael Curry}
\thesissupervisor{Professor Min Chen}
\thesisdepartment{Department of Computer Science\\
  SUNY NEW PALTZ}
\maketitle

% Feb 24, 2010 vandurme> the graduate office wanted a "ii" on the top right
% corner of the dedication page (lowercase Roman numeral page numbering), which
% explains the slightly messy tex code for this page:

%%%dedication page
\thispagestyle{plain}
\newenvironment{dedication}
{\cleardoublepage \vspace*{\stretch{1}}
  \begin{center} \em}
  {\end{center} \vspace*{\stretch{3}} }
\begin{dedication}

   To Damien and Colette

\end{dedication}

\tableofcontents
\listoftables
\listoffigures

%%% CV page
\begin{curriculumvitae}

  Previous degrees and experience.

\end{curriculumvitae}

\begin{acknowledgments}

  Thanks to collaborators and supporters.

  %% Additional personal acknowledgments ...

 
\end{acknowledgments}

\begin{abstract}

  Since the advent of photography practioners have been searching for processed to maximize the detail their images contain.  The photography image is inherneitly a limited representation of our reality.  The discrete nature and the limited technical range condent the visual relationships of our world. Each image sacrifices certain elements to produce a generalized view of the photographers eye. 

Photography in it's earliest form was a practive in capturng brightness values.  It's invention in the 1800's as a Black and White medium was our first semi-permanent (all chemical based and ink based photographs fade over time) mechinacal/chemical process to capture our visual existance. 

Overfitting is a major issue with a limited dataset.  The best CNN models come from big data.  The more images available the better the ability of the model to form a more generalized view of the relationships in the data.  

Image issues: Limited size, lighting, exposure, viewpoint, occlusion, background, scale, \dots

My Thesis will focus on lighting and exposure issues.

Maximize the information in the dataset by creating a more generalized representation by training on the full dynamic range of the image.  
  
\end{abstract}




\chapter{Introduction}

% \chapter{Latex Tips}

\section{Citations}

This template uses the \verb|natbib|
package.
Use the command \verb|\cite| for citations in parentheses.
Use the command \verb|\citet| for citations in text.
Use the command \verb|\citeyearpar| for the year only, in parentheses.
For example,
\begin{verbatim}
    ... as in \LaTeX\ \cite{Lamport86} ...
    ... and \citet{Knuth86} claims ...
    ... and Knuth's later work \citeyearpar{Knuth86} claims ...
\end{verbatim}
will result in
\begin{quote}
... as in \LaTeX\ \cite{Lamport86} ...\\
... and \citet{Knuth86} claims ...\\
... and Knuth's later work \citeyearpar{Knuth86} claims ...
\end{quote}

You can change the template to use another citation style if you prefer.  The only requirement is that citations appear in the style ``accepted in your field.''

\section{Math}

Use \verb|\log| and \verb|\exp|, not \verb|log| and \verb|exp|.

Blank lines start a new paragraph - don't start
a new paragraph after an equation in the middle of a sentence.
Use
\begin{verbatim}
\[ e^{\pi i} = -1 \]
where $i=\sqrt{-1}.$ 
\end{verbatim}
NOT
\begin{verbatim}
\[ e^{\pi i} = -1 \]

where $i=\sqrt{-1}.$ 
\end{verbatim}
in order to avoid having the word ``where'' indented as the
beginning of a new paragraph.

Use \verb|\left(| and \verb|\right)| to get parens that are the right size for whatever is inside them.

For a variable or function name consisting of more than one letter, use \verb|\mathit{func}| or \verb|\mathrm{func}|.  Otherwise, latex interprets this as $f*u*n*c$.

For angled brackets to denote tuples, use \verb|\langle| and \verb|\rangle|, not \verb|<| and \verb|>|.

\section{Text}

TeX assumes that a period ends a sentence unless it follows an uppercase letter.
Use \verb|Smith et al.\ claim|, not \verb|Smith et al. claim|.
At the end of a sentence, use \verb|consisting of an NP\@.|, not \verb|consisting of an NP.|.

``et al.'' is ``et al.'', not ``et. al.'' or ``et. al''

\chapter{Conclusion}

\bibliographystyle{plainnat}
\bibliography{references}

\appendix
\chapter{More stuff}


\end{document}



