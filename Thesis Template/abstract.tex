% $Log: abstract.tex,v $
% Revision 1.1  93/05/14  14:56:25  starflt
% Initial revision
% 
% Revision 1.1  90/05/04  10:41:01  lwvanels
% Initial revision
% 
%
%% The text of your abstract and nothing else (other than comments) goes here.
%% It will be single-spaced and the rest of the text that is supposed to go on
%% the abstract page will be generated by the abstractpage environment.  This
%% file should be \input (not \include 'd) from cover.tex.
Since the advent of photography practioners have been searching for processed to maximize the detail their images contain.  The photography image is inherneitly a limited representation of our reality.  The discrete nature and the limited technical range condent the visual relationships of our world. Each image sacrifices certain elements to produce a generalized view of the photographers eye.  Photography in it's earliest form was a practive in capturng brightness values.  It's invention in the 1800's as a Black and White medium was our first semi-permanent (all chemical based and ink based photographs fade over time) mechinacal/chemical process to capture our visual existance. Overfitting is a major issue with a limited dataset.  The best CNN models come from big data.  The more images available the better the ability of the model to form a more generalized view of the relationships in the data.  Image issues: Limited size, lighting, exposure, viewpoint, occlusion, background, scale, \dots My Thesis will focus on lighting and exposure issues. Maximize the information in the dataset by creating a more generalized representation by training on the full dynamic range of the image.  