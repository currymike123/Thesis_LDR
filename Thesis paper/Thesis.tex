Since the advent of photography practioners have been searching for processed to maximize the detail their images contain.  The photography image is inherneitly a limited representation of our reality.  The discrete nature and the limited technical range condent the visual relationships of our world. Each image sacrifices certain elements to produce a generalized view of the photographers eye. 

Photography in it's earliest form was a practive in capturng brightness values.  It's invention in the 1800's as a Black and White medium was our first semi-permanent (all chemical based and ink based photographs fade over time) mechinacal/chemical process to capture our visual existance. 

Overfitting is a major issue with a limited dataset.  The best CNN models come from big data.  The more images available the better the ability of the model to form a more generalized view of the relationships in the data.  

Image issues: Limited size, lighting, exposure, viewpoint, occlusion, background, scale, \dots

My Thesis will focus on lighting and exposure issues.

Maximize the information in the dataset by creating a more generalized representation by training on the full dynamic range of the image.  